%%%  Ukázkový text a dokumentace stylu pro text závěrečné (bakalářské a
%%%  diplomové) práce na KI PřF UP v Olomouci
%%%  Copyright (C) 2012 Martin Rotter, <rotter.martinos@gmail.com>
%%%  Copyright (C) 2014 Jan Outrata, <jan.outrata@upol.cz>


%%  Pro získání PDF souboru dokumentu je třeba tento zdrojový text v
%%  LaTeXu přeložit (dvakrát) programem pdfLaTeX.

%%  V případě použití programu BibLaTeX pro tvorbu seznamu literatury
%%  je poté ještě třeba spustit program Biber s parametrem jméno
%%  souboru zdrojového textu bez přípony a následně opět (dvakrát)
%%  přeložit zdrojový text programem pdfLaTeX.

%%  Postup získání Postscriptového souboru je popsán v dokumentaci.


%%  Třída dokumentu implementující styl pro závěrečnou práci. Vybrané
%%  nepovinné parametry (ostatní v dokumentaci):

%%  'master' pro sazbu diplomové práce, jinak se sází bakalářská práce

%%  'program=kód' pro Váš studijní program/obor (specializaci), kódy
%%  pro diplomovou práci 'infoi' pro Informatiku (Obecná informatika),
%%  'infui' pro Informatiku (Umělá inteligence), 'ainfpst' pro
%%  Aplikovanou informatiku (Počítačové systémy a technologie), 'uinf'
%%  pro Učitelství informatiky pro střední školy, 'binf' pro
%%  Bioinformatiku, 'inf' pro Informatiku (bez specializací) a 'ainf'
%%  pro Aplikovanou informatiku (bez specializací), jinak je výchozí
%%  ainfvs pro Aplikovanou informatiku (Vývoj software), a pro
%%  bakalářskou práci 'infoi' pro Informatiku (Obecná informatika),
%%  'itp' pro Informační technologie v prezenční formě, 'itk' pro
%%  Informační technologie v kombinované formě, 'infv' pro Informatiku
%%  pro vzdělávání, 'binf' pro Bioinfomatiku, 'inf' pro Informatiku
%%  (bez specializací), 'ainfp' pro Aplikovanou informatiku (bez
%%  specializací) v prezenční formě, 'ainfk' pro Aplikovanou
%%  informatiku (bez specializací) v kombinované formě, jinak je
%%  výchozí infpvs pro Informatiku (Programování a vývoj software)

%%  'printversion' pro sazbu verze pro tisk (nebarevné logo a odkazy,
%%  odkazy s uvedením adresy za odkazem, ne odkazy do rejstříku),
%%  jinak verze pro prohlížeč

%%  'biblatex' pro zapnutí podpory pro sazbu bibliografie pomocí
%%  BibLaTeXu, jinak je výchozí sazba v prostředí thebibliography

%%  'language=jazyk' pro jazyk práce, jazyky english pro anglický,
%%  slovak pro slovenský, jinak je výchozí czech pro český

%%  'font=sans' pro bezpatkový font (Iwona Light), jinak je výchozí
%%  serif pro patkový (Latin Modern)

%%  'figures, tables, theorems a sourcecodes' pro sazbu seznamu
%%  obrázků, tabulek, vět a zdrojových kódů, jinak při =false se
%%  nesází (u theorems a sourcecodes výchozí)

\documentclass[
%  master,
%  program=ainfvs,
%  printversion,
%  biblatex,
%  language=english,
%  font=sans,
  figures=false,
%  tables=false,
%  theorems,
%  sourcecodes,
  glossaries,
  index
]{kidiplom}

%% Informace pro úvodní strany. V jazyku práce (pokud není v komentáři
%% uvedeno česky) a anglicky. Uveďte všechny, u kterých není v
%% komentáři uvedeno, že jsou volitelné. Při neuvedení se použijí
%% výchozí texty. Text pro jiný než nastavený jazyk práce (nepovinným
%% parametrem language makra \documentclass, výchozí český) se zadává
%% použitím makra s uvedením jazyka jako nepovinného parametru.

%% Název práce, česky a anglicky. Měl by se vysázet na jeden řádek.
\title{Webová aplikace s texty Bible}
\title[english]{Web application with Bible texts}

%% Volitelný podnázev práce, česky a anglicky. Měl by se vysázet na
%% jeden řádek. Výchozí je prázdný.
%\subtitle{Ukázkový text a dokumentace stylu v \LaTeX{}u}
%\subtitle[english]{Sample text and documentation of the \LaTeX{} style}

%% Jméno autora práce. Makro nemá nepovinný parametr pro uvedení
%% jazyka.
\author{Patrik Prinz}

%% Jméno vedoucího práce (včetně titulů). Makro nemá nepovinný
%% parametr pro uvedení jazyka.
\supervisor{Mgr. Tomáš Urbanec, Ph.D.}

%% Volitelný rok odevzdání práce. Výchozí je aktuální (kalendářní)
%% rok. Makro nemá nepovinný parametr pro uvedení jazyka.
%\yearofsubmit{\the\year}

%% Anotace práce, včetně anglické (obvykle překlad z jazyka
%% práce). Jeden odstavec!
\annotation{Ukázkový text závěrečné práce na Katedře informatiky
  Přírodovědecké fakulty Univerzity Palackého v Olomouci, který je
  zároveň dokumentací stylu pro text práce v \LaTeX{}u. Zdrojový text
  v \LaTeX{}u je doporučeno použít jako šablonu pro text skutečné
  závěrečné práce studenta.}

\annotation[english]{Sample text of thesis at the \kitextdepten,
  \kitextfacultyen, \kitextuniven{} and, at the same time,
  documentation of the \LaTeX{} style for the text. The source text in
  \LaTeX{} is recommended to be used as a template for real student's
  thesis text.}

%% Klíčová slova práce, včetně anglických. Oddělená (obvykle) středníkem.
\keywords{webová aplikace; Bible; PWA; TypeScript; DevOps; CI/CD}
\keywords[english]{web application; Bible; PWA; TypeScript; DevOps; CI/CD}

%% Volitelná specifikace příloh textu práce, i anglicky. Výchozí je
%% 'elektronická data v systému katedry informatiky / electronic data
%% in system of department of computer science'.
%\supplements{nejlepší software všech dob}
%\supplements[english]{the best software of all times}

%% Volitelné poděkování. Stručné! Výchozí je prázdné. Makro nemá
%% nepovinný parametr pro uvedení jazyka.
\thanks{Děkuji, děkuji, děkuji.}

%% Cesta k souboru s bibliografií pro její sazbu pomocí BibLaTeXu
%% (zvolenou nepovinným parametrem biblatex makra
%% \documentclass). Použijte pouze při této sazbě, ne při (výchozí)
%% sazbě v prostředí thebibliography.
\bibliography{bibliografie.bib}

%% Další dodatečné styly (balíky) potřebné pro sazbu vlastního textu
%% práce.
\usepackage{lipsum}
\usepackage{longtable}

\begin{document}
%% Sazba úvodních stran -- titulní, s bibliografickými údaji, s
%% anotací a klíčovými slovy, s poděkováním a prohlášením, s obsahem a
%% se seznamy obrázků, tabulek, vět a zdrojových kódů (pokud jejich
%% sazba není vypnutá).
\maketitle

%% Vlastní text závěrečné práce. Pro povinné závěry, před přílohami,
%% použijte prostředí kiconclusions. Povinná je i příloha s obsahem
%% elektronických dat.

%% -------------------------------------------------------------------

\newcommand{\BibLaTeX}{\textsc{Bib}\LaTeX}

% \noindent\textcolor{red}{\LARGE Upozornění: Následující text
%   dokumentace stylu, vyjma přílohy~\ref{sec:ObsahData}, je rozpracovaná
%   a (značně) neúplná verze!!!}

\section{Úvod}
Cílem této práce je Vytvořit aplikaci pro použití v Řeckokatolické farnosti v Olomouci. Farnost je v současnosti vícenárodnostní a v posledních letech se značně
rozrůstá. nejpočetněji jsou zde zastoupeni farníci z Ukrajiny. Texty Bible jsou v katolické církvi velmi často využívané při liturgických obřadech, ale také při
osobní modlitbě a rozjímání.

Pro Apoštolský Exarchát České Republiky, který je nadřazenou organizací Olomoucké farnosti existuje kalendář, který obsahuje i biblické texty pro každý den v roce.
Biblické texty texty v církevních obřadech jsou často vázány ke svátkům, které jsou buď pevně svázány s datem, nebo jsou pohyblivé a připadají kyždý rok na jiné datum.
Mnoho textů je také svázáno s konkrétním obřadem, nebo událostí (např. křest, svěcení domu...).

Texty z knihy žalmů jsou často často používány jako samostatné modlitby. Východní církve žalmy rozdělují do 20 skupin zvaných kaftizmy, které obsahují většinou okolo
4-6 žalmů. Ke kaftizmám jsou přidruženy další texty, které se modlí zároveň s nimi. Farnost má vlastní společenství, které se modlí modlitby žalmů a funguje přibližně od
roku 2024. Každý člen společenství se modlí jednu z kaftizem během týdne. Kaftizmy se mezi členy společenství pravidelně střídají.

Nenašel jsem dosud žádný projekt, který by poskytoval možnosti práce s biblickými texty v tomto rozsahu. Proto chci vytvořit aplikaci, která by sdružovala by texty více
jazyků s možností vlastní správy a přidávání dalších překladů a textů pro vlastní četbu a studium. S tímto bude spojena i možnost udržování vlastního kalendáře který
bude obsahovat zejména úryvky na každý den. Také by zde měla být obsažena možnost vytvářet si a ukládat vlastní úryvky. Proces rozdělení kaftizem v rámci společenství
žaltáře plánuji zautomatizovat a zároveň poskytnout na stejném místě i texty samotné i spřidruženými modlitbami spolu s možnostíoznačit, že se člen již kaftizmu pomodlil.

Dále bych chtěl, aby platforma byla do budoucna dále rozšiřitelná a bylo možné přidávat další funkcionality a aby mohl projekt i v budoucnu reflektovat požadavky
společenství, které jej bude využívat.

Dále se chci v této práci zaměřit na průzkum a využití konceptů z oblasti DevOps, které usnadní vývoj samotný, urychlí integraci a nasazení aplikace. Na projektu pracuji sám,
proto je pro mne výhodné především zautomatizovat proces ověření a nasazení software a sledování jeho stavu. I za cenu větší náročnosti přípravy a počáteční konfigurace je pro
mne výhodné získat značně větší spolehlivost a rychlost nasazení a také do budoucna usnadní řadu problémů s udržováním potřebné infrastruktury. Také doufám, že nabyté znalosti z
této oblasti využiji i v budoucnu ve svém profesním životě.
\section{Podobné projekty}
V rámci této práce jsem hledal a procházel i další software zaměřený na práci s texty Bible i dalšími liturgickými texty. Aplikací s tímto zaměřeních existuje celá řada.
Většina je však určena k jiným účelům a předvším pro osobní použití, ne pro použití ve farnosti samotné, případně v jiném společenství. Podobné projekty bývají zaměřeny
na samotné čtení a studium biblických textů. Většina z projektů má také uzavřený zdrojový kód a proto je nelze upravit a přizpůsobit pro specifické potřeby, nebo je již
zastaralá a dlouhodobě neudržovaná. Mnou vyvíjená aplikace má řadu velmi specifických požadavků Proto má  jen malý průnik s řadou

\subsection{}
\section{Struktura projektu}
Projekt je strukturován do monorepozitáře, který obsahuje několik balíčků (packages), do kterých je funkcionalita rozdělena.
Kořenový balíček monorepozitáře obsahuje společnou konfiguraci nástrojů a společné balíčky.

Pro správu jsem použil balíčkovací nástroj pnpm pro node.js, který je alternativou k npm. Pro projekty s větším množstvím závislostí poskytuje značně lepší výkon a také
možnost používat workspaces. Workspaces umožňují mít v rámci projektu více balíčků s vlastní konfigurací a vlastními závislostmi.
\subsection{Balíčky}
\subsubsection{Kořenový balíček}
Kořenový balíček má nastarosti uchovávání konfigurace
\subsection{Nastavení projektu}
Popis Základní struktury projektu, konfigurace a popis složek a souborů v kořenovém balíčku.
\subsection{Repozitář a CI/CD pipeline}


Pro verzování projektu jsem požil GIT a jako vzdálené úložiště GitHub. Nástroj GIT je velmi populární a v současnosti má velmá málo použitelných alternativ. GitHub jsem zvolil táké pro jeho rozšířenost a Řadu možností, které poskytuje. mezi ně patří především služba GitHub Actions, kterou jsem použil pro vytvoření CI/CD pipeline.

GitHub Actions je flexibilní nástroj, který umožňuje spouštět akce, jako reakci na události v repozitáři.



\subsection{Styl kódu}
Popis konvencí
\subsection{Backend}
Sdílené soubory

Aplikace na backendu má modulární monolitickou architekturu, která zajišťuje rozdělení funkcionalit do skupin podle jejich domény a tím usnadňuje i kontrolu závislostí a umožňuje snadnější rozšířitelnost.

Jádrem aplikace je sada globálně dostupných nástrojů, které implementují funkcionality, které jsou nezávislé na doméně a používají se na více místech napříč aplikací.

Patří sem například middleware, který zajišťuje valitaci dat, ošetření chyb a logování. Dále také sdílené knihovny, jako například databázový adaptér.

Moduly
\subsubsection{Struktura modulu}
Moduly sdružují funkcionalitu, která je spojena na úrovni domény projektu.
Každý modul obsahuje:
\begin{itemize}
	\item \textbf{router:} mapuje HTTP požadavky na příslušné metody controlleru
	\item \textbf{controller:} Validuje a zpracovává data a požadavky. Zajišťuje komunikacis routerem a spouští service
	\item \textbf{service:} zajišťuje získání a zpracování dat na základě požadavku. Výsledek vrací controlleru. Zajišťuje komunikaci mezi jednotlivými použitými závislostmi. Uchovává závislosti
	\item \textbf{repository:} zajišťuje komunikaci s databází prostřednictvím adaptéru a poskytuje metody specifické pro danou doménu. Provádí překlad mezi doménovými typy a typy pro elasticsearch klienta. Zajišťuje kontrolu a validaci dat. Obstarává konzistentnost dat v databázi.
\end{itemize}

\subsection{Frontend}
\subsection{Databáze}
\section{Použité technologie}
\subsection{TypeScript}
\subsection{Express}
\subsection{Elastic search}
\subsection{Docker}
\subsection{Github actions}

%% -------------------------------------------------------------------

%% Sazba volitelného seznamu zkratek, za přílohami.
\printglossary

%% Sazba povinné bibliografie, za přílohami (případně i za seznamem
%% zkratek). Při použití BibLaTeXu použijte makro
%% \printbibliography. jinak prostředí thebibliography. Ne obojí!

%% Sazba i v textu necitovaných zdrojů, při použití
%% BibLaTeXu. Volitelné.
\nocite{*}
%% Vlastní sazba bibliografie při použití BibLaTeXu.
%\printbibliography

%% Bibliografie, včetně sazby, při NEpoužití BibLaTeXu.
% \begin{thebibliography}{9}
%\bibitem{kniha2} \uppercase{Hawke}, Paul. NanoHttpd: Light-weight HTTP server designed for embedding in other applications. GitHub [online]. 2014-05-12. [cit. 2014-12-06]. Dostupné z: \url{https://github.com/NanoHttpd/nanohttpd}
%
%\bibitem{jeske13} \uppercase{Jeske}, David; \uppercase{Novák}, Josef. Simple HTTP Server in \csharp: Threaded synchronous HTTP Server abstract class, to respond to HTTP requests. CodeProject: For those who code [online]. 2014-05-24. [cit. 2014-12-06]. Dostupné z: \url{http://www.codeproject.com/Articles/137979/Simple-HTTP-Server-in-C}
%
%\bibitem{uzis2012} \uppercase{ÚSTAV ZDRAVOTNICKÝCH INFORMACÍ A STATISTIKY ČR}. Lékaři, zubní lékaři a farmaceuti 2012 [online]. Praha 2, Palackého náměstí 4: Ústav zdravotnických informací a statistiky ČR, 2012 [cit. 2014-12-06]. ISBN 978-80-7472-089-5. Dostupné z: \url{http://www.uzis.cz/publikace/lekari-zubni-lekari-farmaceuti-2012}
% \end{thebibliography}

%% Sazba volitelného rejstříku, za bibliografií.
\printindex

\end{document}

%%% Local Variables:
%%% mode: latex
%%% TeX-master: t
%%% End:
